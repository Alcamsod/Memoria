\section*{Resumen}

Durante mis prácticas en Evenbytes S.L., una consultora tecnológica especializada en soluciones cloud y transformación digital, he desarrollado competencias, de manera principal, en ciencia de datos. En la fase inicial, centrada en la formación acerca del flujo de trabajo propio de la empresa, utilicé el \textit{framework} Angular, el entorno de \textit{Google Cloud Platform} (\textit{GCP}) y el sistema de control de versiones \textit{GitHub}. Posteriormente, participé en diversos proyectos, destacando especialmente uno orientado a la implementación de un modelo de predicción del precio de la electricidad y a la optimización de los procesos de fundición de unos hornos para la reducción del consumo energético. Esta experiencia me ha permitido integrar los conocimientos adquiridos en los grados de Matemáticas y Física, además de proporcionarme una visión más clara y aplicada del entorno profesional tecnológico.

\vspace{1cm}
\noindent\textbf{Palabras clave:} Ciencia de datos, modelo de predicción, machine learning, Google Cloud Platform
 

\section*{Abstract}

During my internship at Evenbytes S.L., a technology consulting firm specialized in \textit{cloud} solutions and digital transformation, I primarily developed skills in data science. In the initial phase, focused on learning the company's workflow, I used the Angular \textit{framework}, the \textit{Google Cloud Platform} (\textit{GCP}) environment, and the \textit{GitHub} version control system. Subsequently, I participated in various projects, most notably one aimed at implementing a model for electricity price prediction and optimizing furnace casting processes to reduce energy consumption. This experience has allowed me to integrate the knowledge acquired during my Mathematics and Physics degrees, in addition to providing me with a clearer and more applied perspective of the professional tech environment.

\vspace{1cm}
\noindent\textbf{Keywords:} Data science,  predictive model, machine learning, Google Cloud Platform

\clearpage
% Índices
\hypersetup{linkcolor=black}
\tableofcontents\clearpage
\listoffigures\clearpage
% \listoftables\clearpage
\hypersetup{linkcolor=linkscolor}

\chapter*{Siglas y acrónimos}
\addcontentsline{toc}{chapter}{Siglas y acrónimos}
%
%
Como es habitual en documentos técnicos, a menudo se emplean sigla y acrónimos para abreviar términos específicos del área. Más aún, en el ámbito científico, el inglés predomina como lengua de trabajo por lo que muchos términos se encuentran en este idioma y la traducción al español no es siempre directa. A continuación se presenta una lista de los acrónimos más relevantes junto con su significado:
%
%
\begin{description}
    \item[UC] Universidad de Cantabria
    \item[IA] Inteligencia artificial
    \item[TFT]  Temporal Fusion Transformers (Transformadores de fusión temporal)
    \item[XGBoost] Extreme Gradient Boosting (Aumento extremo de gradiente)
    \item[SARIMAX] (Seasonal AutoRegressive Integrated Moving Average with eXogenous regressors) (Medias móviles integradas autorregresivas estacionales con regresores exógenos)
    \item[GCP] Google Cloud Platform (Plataforma Google Cloud)
    \item[TSF] Time series forecasting (Predicción de series temporales)
    \item[MAE] Mean absolute error (error absoluto promedio)
    \item[MAE] Root mean squared error (raiz del error cuadrático promedio)
    \item[ACF] Autocorrelation function (función de autocorrelación)
    \item[PACF] Partial autocorrelation function (función de autocorrelación parcial)
    \item[API] Application Programming Interface (Interfaz de Programación de Aplicaciones)
    \item[PVPC] Precio de venta al pequeño consumidor 
    \item[ESIOS] Estructura de Intercambio de Información del Operador del Sistema (Operador del Sistema Eléctrico Español)  
\end{description}

\chapter*{Agradecimientos}
\addcontentsline{toc}{chapter}{Agradecimientos}
%
%
Me gustaría agradecer a todas las personas de Evenbytes su amabilidad e integración, la plantilla de trabajo es como una gran familia y han hecho que este tiempo haya sido muy agradable. En particular, a Miguel y Pedro, quienes más cerca de mí han estado en ese inmejorable \textit{equipo Data} y que me han enseñado día a día. Con ellos me he divertido, he aprendido y he cumplido uno de mis objetivos con estas práticas, que era conocer de primera mano el mundo empresarial. Naturalmente, agradecer también a Jacinto Pelayo, CEO de la empresa, su confianza desde el primer momento para la incoporación a la empresa. Finalmente, a mi tutora académica, Alicia Nieto, por haberme ayudado en la organización y elaboración de esta memoria.

En útimo lugar, no puedo dejar pasar la oportunidad (ni me dejarían) de agradecer a mi familia su apoyo. Como es comprensible, el inicio en el mundo laboral no es sencillo y gracias a ellos me ha costado menos.

En resumen, estas prácticas no solo me han permitido aplicar conocimientos adquiridos a lo largo de estos cuatro años de formación, sino que también han sobrepasado las expectativas de conocer el mundo empresarial y me han proporcionado aquella visión que necesitaba para decidir mi futuro profesional.

\begin{flushright}
    \textit{Muchas gracias}
\end{flushright}