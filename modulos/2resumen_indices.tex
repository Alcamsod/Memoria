\section*{Resumen}

Durante mis prácticas en Evenbytes S.L., una consultora tecnológica especializada en soluciones cloud y transformación digital, he desarrollado competencias en el diseño de aplicaciones web y, de manera principal, en ciencia de datos. En la fase inicial, centrada en el desarrollo web, trabajé con el \textit{framework} Angular, el entorno de Google Cloud Platform (\textit{GCP}) y el sistema de control de versiones \textit{GitHub}. Posteriormente, participé en diversos proyectos, destacando especialmente uno orientado a la implementación de un modelo de predicción del precio de la electricidad y a la optimización de los procesos de fundición de unos hornos para la reducción del consumo energético. Esta experiencia me ha permitido integrar los conocimientos adquiridos en los grados de Matemáticas y Física, además de proporcionarme una visión más clara y aplicada del entorno profesional tecnológico.

\vspace{1cm}
\noindent\textbf{Palabras clave:} Ciencia de datos, Optimización, Modelos de predicción, Python, GPC.
 

\section*{Abstract}

During my internship at Evenbytes S.L., a technology consulting firm specializing in cloud solutions and digital transformation, I developed skills in web application design and, primarily, in data science. In the initial phase, focused on web development, I worked with the Angular framework, the Google Cloud Platform (GCP) environment, and the GitHub version control system. Later, I participated in various projects, particularly one aimed at implementing a model for predicting electricity prices and optimizing the smelting processes of furnaces to reduce energy consumption. This experience has allowed me to integrate knowledge acquired in Mathematics and Physics degrees, as well as providing me with a clearer and more applied view of the technological professional environment.

\vspace{1cm}
\noindent\textbf{Keywords:} Data science, Optimization, Predictive models, Python, GPC.

\clearpage
% Índices
\hypersetup{linkcolor=black}
\tableofcontents\clearpage
\listoffigures\clearpage
\listoftables\clearpage
\hypersetup{linkcolor=linkscolor}

\chapter*{Siglas y acrónimos}
\addcontentsline{toc}{chapter}{Siglas y acrónimos}
%
%
\begin{description}
    \item[UC] Universidad de Cantabria
    \item[GCP] Google Cloud Platform (Plataforma Google Cloud)
    \item[TSF] Time series forecasting (Predicción de series temporales)
    \item[TFT] Time series forecasting transformer (Transformador de predicción de series temporales) 
    \item[SARIMAX] (Seasonal AutoRegressive Integrated Moving Average with eXogenous regressors) (Medias variables integradas autorregresivas estacionales con regresores exógenos)
    \item[CEO] Chief Executive Officer (Director ejecutivo)
    \item[CTO] Chief Technology Officer (Director tecnológico)
    \item[COO] Chief Operating Officer (Director de operaciones)
    \item[URL] Uniform Resource Locator ( Localizador Uniforme de Recursos)
    \item[gif] Graphics Interchange Format (Formato de Intercambio de Gráficos) \footnote{ Según la RAE, \textit{aunque GIF sea una siglas, la RAE lo considera ya un acrónimo lexicalizado (una palabra a todos los efectos), con minúsculas: «gif», «gifs» (en cursiva, por tratarse de una voz no adaptada, ya que en español a esa grafía le correspondería la pronunciación [jíf])} \cite{rae2018tweet}.} 
    \item[CSV] Comma separated values (valores separados por comas).
    \item[IA] Inteligencia artificial
    \item[ACF] Autocorrelation function (función de autocorrelación)
    \item[PACF] Partial autocorrelation function (función de autocorrelación parcial)
    \item[API] Application Programming Interface (Interfaz de Programación de Aplicaciones)
    \item[ESIOS] Estructura de Intercambio de Información del Operador del Sistema (Operador del Sistema Eléctrico Español)  
\end{description}

\chapter*{Agradecimientos}
\addcontentsline{toc}{chapter}{Agradecimientos}
[Texto opcional de agradecimientos.]