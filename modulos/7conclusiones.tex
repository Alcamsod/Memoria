\section{Relación entre formación recibida y actividades}
\label{Relacion entre formacion recibida y actividades}
%
%
%
Con salvedad de la etapa de formación, creo que ha quedado de manifiesto la íntima relación entre los conocimientos adquiridos durante el grado en matemática, y con menor relevancia en física, y las tareas desempeñadas durante este tiempo. Intentando particularizar en asignaturas: Programación, impartida durante el segundo cuatrimestre del primer curso, las asignaturas de estadística, principalmente Inferencia estadística, del segundo cuatrimestre de tercero (cuarto en mi caso debido a la distribución del Doble Grado) y en cuanto a física, cláramente, Termodinámica.

Aunque a priori pueda parecer que no hay mucha relación entre modelos de redes neuronales, machine learning y deep learning, una vez investigas y te adentras en los mecanismos que hacen que estas tecnologías (bases de la IA) funcionen, rápidamente observas que la estadística son clave en estas áreas. Esta característica, que quizás queda algo más alejada del público general, es uno de los mayores atractivos para personas con formación en matemáticas que se sienten atraídas por una de la que probablemente sea la revolución del siglo: La IA.
%
%
\section{Atención y asesoramiento recibido} \label{Atencion y asesoramiento recibido}
%
%
Durante mi estancia en la empresa el trato y atención han sido inmejorables. En este tiempo me he encontrado bajo la supervisión de mis dos compañeros del equipo de Data, ambos analistas de datos, si bien cabe remarcar que el apoyo global de la plantilla ha estado presente.

En todo el periodo he estado presente y participado activamente en reuniones de proyectos, tomas de decisión junto a los responsables (CEO,CTO y COO), de manera que además del aprendizaje técnico he sido capaz de conocer el funcionamiento de situaciones habituales en el entorno laboral.
%
%
\chapter{Conclusiones}
%
%
El programa de prácticas llevado a cabo estos dos meses ha sido excelente, más que satisfactorio. Como se ha comentado en la introducción y varias veces a lo largo de esta memoria, mi principal interés estaba en el acercamiento al mundo laboral, de manera que pudiera conocer más de cerca el sector. En estos dos meses y medio me he adentrado dentro de un mundo que desconocía completamente más allá del interés superficial que suscita la IA y sus aplicaciones a situaciones de mercado.

Todas las habilidades, tanto técnicas como personales, adquiridas en este tiempo serán de una enorme utilidad para toda la vida. Además, el aprendizaje llevado a cabo en tareas como el trabajo en equipo, los plazos de entrega, las decisiones en proyectos y el trato con clientes han permitido que tenga una primera toma de contacto con el mundo profesional.

A modo de resumen y para finalizar: Las primeras semanas, en las que mi formación estuvo enfocada en un ámbito que desconocía totalmente y mi posterior inclusión en un grupo de trabajo ajeno tanto a la formación como a mi conocimiento, supusieron unas semanas de dureza ya que todo era nuevo. Progresivamente fui aprendiendo y adaptandome al flujo de trabajo, viendome incluido en diversos proyectos, entre ellos el principal expuesto aquí, de gran dificultad tanto en lo técnico como en su implementación para el cliente. Sin embargo, esta situación me ha enseñado a trabajar de manera autónoma, documentarme y asumir el liderazgo para la toma de decisiones, confiando en mi trabajo y conocimientos, de manera que he podido sacar adelante el trabajo.
%
%
%