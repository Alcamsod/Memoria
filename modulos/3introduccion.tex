\chapter{Introducción}
%
%
%
\section{Elección} \label{eleccion}
%
%
El proceso que llevó a la decisión de incorporarme a Evenbytes estuvo influenciado por diversos factores. En primer lugar, tuve la fortuna de que amigos míos, ya graduados de esta misma facultad, realizasen las prácticas en esta empresa, teniendo estos además diferentes perfiles, de manera que contaba de inicio con la ventaja de conocer la empresa de manera previa. Además, durante la feria de empleo realizada el 30 de abril de este mismo año pude asistir a los diferentes puestos que situaban las empresas asistentes para promocionarse y captar posibles futuros trabajadores. Dentro de estos grupos, Evenbytes fue la empresa que más llamó mi atención gracias a un proyecto que involucraba la implementación de predicciones estadísticas, modelado físico y optimización matemática. Fue por tanto la conjunción de un proyecto que aunase los conocimientos de las dos carreras en las que me estoy formando, con conocer de manera previa el buen ambiente de trabajo, la que hizo que me decantase por Evenbytes.
%
%
\section{Objetivos} \label{objetivos}
%
%
La decisión acerca del futuro laboral es una tarea complicada, y más aún cuando tu formación te otorga una plenitud de opciones. Además, me encontraba ante la indecisión de si continuar por el mundo académico o desligarme de lo \textit{teórico} e insertarme en la aplicación de los conocimientos adquiridos. Ante esta situación, veía más que necesario tener un contacto directo con el mundo profesional para poder dilucidar cuál quería que fuera mi futuro. Entre las ramas de mi interés se encontraban principalmente el sector financiero y el tecnológico, ya que considero que ambas áreas ofrecen un campo de desarrollo muy amplio para los perfiles con formación en Física y Matemáticas.

Por tanto, buscar una empresa que encajase con estas características era indispensable. La mayor parte de las ofertas actuales, especialmente en ámbitos relacionados con el análisis de datos, la modelización estadística y el desarrollo de soluciones tecnológicas, se alinean con esta línea de interés. A la hora de seleccionar un centro que me aceptase, una de mis prioridades fue también el ambiente laboral, puesto que, en mi opinión, uno aprende más cuando se encuentra cómodo.

En este aspecto, Evenbytes me acogió de manera total, siendo un claro ejemplo de un entorno de trabajo saludable en el que se valora la iniciativa, la capacidad de adaptación y la aportación personal al proyecto común. Considero que realizar las prácticas en una empresa con estas características me permitía no sólo adquirir experiencia práctica sino también contrastar en primera persona cómo se aplican competencias transversales como el pensamiento crítico, toma de decisiones en proyectos reales y la proactividad. Asimismo, esta oportunidad suponía un primer paso importante para explorar con mayor claridad qué caminos profesionales me gustaría recorrer en el futuro.
%
%
\section{Centro} \label{centro}
%
%
Evenbytes es una consultora tecnológica que impulsa la transformación digital de empresas mediante soluciones tecnológicas avanzadas. Con un recorrido de más de 10 años de experiencia, su objetivo es ayudar a las organizaciones a ser más eficientes, competitivas e innovadoras, adaptándose siempre a sus necesidades específicas.

Especializados en el desarrollo de software, infraestructuras en la nube y consultoría tecnológica, en Evenbytes se posiciona como un aliado estratégico para empresas de todos los tamaños, desde startups hasta grandes corporaciones. Son Google Cloud Partners y cuentan con experiencia consolidada en áreas como IA, Big Data e IoT, poniendo estas tecnologías al alcance de sus clientes para que puedan afrontar los retos de un entorno empresarial en constante evolución. Sus principales áreas de especialización incluyen el desarrollo de software personalizado, gestión de infraestructuras en la nube, automatización y optimización de procesos, Inteligencia artificial y Big Data e integración de sistemas y plataformas.

La empresa actualmente cuenta con 16 trabajadores y 4 estudiantes de prácticas. Entre los perfiles de los compañeros se encontraban principalmente el Grado universitario en Ingeniería Informática, Ingeniería de Telecomunicaciones, Grados superiores relacionados con la informática y Grado en Matemáticas. La actividad está dividida en áreas de gerencia, ventas, consultoría, desarrollo y ciencia de datos. Los 3 socios y principales responsables ocupan los cargos de CEO, CTO y COO.

%
%
\section{Práctica Externa} \label{practica externa}
%
%
Las labores realizadas durante el tiempo de prácticas han sido diversas. Como bien es sabido, la formación tanto en matemáticas como en física dotan de una gran capacidad de polivalencia a la persona que la posee, por lo que los trabajos que he realizado han sido diversos. Previamente se ha mencionado que Evenbytes es una empresa que se encuentra en plena expansión dentro del sector de la transformación digital. Por ello, en las semanas iniciales de las prácticas llevé a cabo labores de formación relacionadas principalmente con esta rama: Programación en Angular, diseño de páginas web, entorno GCP, GitHub, etc. Tras haber realizado dichas tareas me incorporé en el equipo de \textit{Data Science}. En este departamento fui incluido en varios proyectos que se describirán a continuación, en los que he aprendido diversos programas relacionados con la ciencia de datos como \textit{BigQuery, Jupyter, Python ...} además de técnicas para el tratado de datos y predicciones como \textit{SARIMAX o redes neuronales}. Adicionalmente, debido a las características de uno de los proyectos me vi involucrado en el modelado físico de un proceso termodinámico, por lo que también he puesto en práctica los conocimientos adquiridos en el grado en Física.